\documentclass{article}
\usepackage{hyperref}
\usepackage{amsmath}

\begin{document}
	
	Given a set of rings with radius $R$ at varying $z$ values, the $B$-field at $z'$ on the $z$-axis due to a ring at $z^*$ is
	\[
		\int_{z^*}^{z'} \frac{\mu_0 Idl}{4\pi} \frac{R}{(z^2 + R^2)^\frac{3}{2}} = 
		\frac{\mu_0}{4\pi} \frac{2 \pi R^2 I}{(z^2 + R^2)^\frac{3}{2}}\Biggr|_{z^*}^{z'}
	\] 
	By the Biot-Savart law.
	
	Thus for a solenoid with $N$ loops, approximated as a series of current carrying loops where each loop is at some value $z_i$ where $0 < i < N$, The net $B$-field at a point $z$ will be
	\[
		\sum_{i=0}^{i=N} 
		\frac{\mu_0}{4\pi} \frac{2 \pi R^2 I}{((z)^2 + R^2)^{3/2}} -
		\frac{\mu_0}{4\pi} \frac{2 \pi R^2 I}{((z^*)^2 + R^2)^{3/2}}
		= B_z
	\]
	
	A derivation of the Amperian model for forces between magnetic dipoles in non-uniform magnetic fields can be found \href{www.phys.ufl.edu/~acosta/phy2061/lectures/MagneticDipoles.doc}{here}, which states:
	\[F_z = \mu \frac{\delta B_z}{\delta z}\]
	For point-like dipole. For a very thin, bar magnet with length $L$, the Force would be:
	\[F_z = \mu \frac{\Delta B_z}{\Delta z} = \mu \frac{\Delta B_z}{L}\]
	
	We can determine the overall change in kinetic energy, $K$, by numerically approximating the following. Note: z is the location of the end of the magnet.
	\[
		\Delta K = -\Delta U = \int_{z_0}^{z} F_z dz = 
		\int_{z_0}^{z} \mu \frac{B(z) - B(z - L)}{L}
	\]
	
	And thus we can determine the speed at which we expect a dipole in our coil gun to travel from its starting position, $z_0$ to the point $z$.
	
	To determine the moment, $\boldsymbol{\mu}$, of the magnet, we approximate the magnet as a series of looped coils with moving charge. Given a constant charge density, $J = dI/dx$, we can use the Biot-Savart law to derive that that the magnetic field  strength at the end of a magnet, $B_{end}$.    
	$$ dI = Jdx $$
	$$ 
	B_{end} = \int_{x=0}^{x=L} \frac {\mu_0 R^2 Jdx} {2(R^2 + x^2)^{3/2}} =
			  \frac {\mu_0J} {2} \frac {L} {\sqrt{L^2 + R^2}}
	$$
	
	Solving for the charge density and the total current moving in the magnet gives
	$$ J = B_{end} \frac{\sqrt{L^2 + R^2}}{L} \frac{2}{\mu_0} $$
	$$ I = JL $$
	$$ I = B_{end} \frac{2 \sqrt{L^2 + R^2}}{\mu_0} $$
	
	Thus as $\boldsymbol{\mu} = AI_{total}$,
	
	$$ \boldsymbol{\mu} = \pi R^2 B_{end} \frac{2 \sqrt{L^2 + R^2}}{\mu_0} $$
	
	For a given magnet with a field strength B_end.
	

\end{document}